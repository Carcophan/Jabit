  \section{Einführung}

  \subsection{Was sind Metadaten?}

  Verschlüsselungstechnologie wie PGP oder S/MIME ermöglicht es zwar auf sichere Art und Weise Nachrichten vor neugierigen Augen zu schützen, doch seit Edward Snowden den NSA-Skandal aufgedeckt hat wissen wir dass Metadaten --- vor allem Informationen darüber wer mit wem kommuniziert -- genauso interessant und viel einfacher zu analysieren sind.

  Es gibt einige Beispiele wie Sie durch Metadaten in Schwierigkeiten kommen können. Wenn Sie jemandem schreiben der in der IS ist, kann es durchaus sein dass Sie das nächste mal nicht in die USA fliegen können. Die No-Fly-Liste kümmert sich nicht darum dass Sie Journalist sind, oder keine Ahnung hatten dass diese Person ein Terrorist war.

  Wenn Samsung erfährt, dass Apple ergiebig mit dem einzigen Produzent eines raffinierten kleinen Sensors ist, brauchen sie keine Details --- das S7 wird ebenfalls einen solchen enthalten. (Dass Apple ihn braucht um ein Auto zu bauen haben sie dabei übersehen.)

  \subsection{Wie können wir Metadaten verstecken?}

  Mit E-Mail können wir die Verbindung zu unserem Mail-Provider verschlüsseln, und dieser wiederum die Verbindung mit dem Provider unseres Gesprächspartners. Dabei können wir nur hoffen dass unser Anbieter und derjenige des Empfängers sowohl vertrauenswürdig als auch kompetent sind.\footnote{Gratis sollte er natürlich auch noch sein.}

  Bei Bitmessage senden wir eine Nachricht and eine grosse Anzahl Teilnehmer, darunter den eigentlichen Empfänger. Die Nachricht ist dabei so verschlüsselt, dass nur die Person welche den privaten Schlüssel besitzt diese entschlüsseln kann. Alle Teilnehmer versuchen dies um die für sie bestimmten Nachrichten zu finden.