\documentclass{bfh}

\usepackage[numbers]{natbib}

\title{Informatikseminar}
\subtitle{Bitmessage -- Communication Without Metadata}
\author{Christian Basler}
\tutor{Kai Brünnler}
\date{\today}

\newcommand{\msg}[1]{\textit{#1}}
\newcommand{\obj}[1]{\textbf{#1}}
\newcommand{\node}[1]{\textbf{#1}}

\begin{document}
  \maketitle

  \tableofcontents

  \section{Synopsis}

  TODO


  % Section basics
    \section{Basics}

  While encryption technology like PGP or S/MIME provides a secure way to protect content from prying eyes, we lerned from Edward Snowden that metadata - information about who communicates with whom - is equally interesting and much easier to analyze.

  With e-mail, we can only prevent this by encrypting the connection to the server as well as between servers. Therefore we can only hope that both our and the recipient's e-mail provider are both trustworthy and competent.

  With Bitmessage we send a message to a sufficiently large number of participants, with the intended recipient among them. The message is encrypted such as only the person in possesion of the private key can decrypt it. All participants try to do this in order to find their messages.


  \section{Protocol}

  \subsection{Nomenclature}

  There are a few terms that are easily mixed up. Here's a list of the most confusing ones:

  \listinginfo{}{message}{is sent from one node to another, i.e. to announce new objects or to initialize the network connection.}{}
  \listinginfo{}{msg}{is the object payload containing the actual message written by a user. The term 'message' is never used to describe information exchange between users in this document. 'Content' is mostly used instead.}{}
  \listinginfo{}{payload}{There are two kinds of payload: message payload for message types, e.g. containing inventory vectors, and object payload, which is distributed throughout the network.}{}
  \listinginfo{}{object}{is a kind of message whose payload is distributed among all nodes. Somtimes just the payload is meant.}{}

  \subsection{Process Flow}

  The newly started node \node{A} connects to a random node \node{B} from its node registry and sends a \msg{version} message, announcing the latest supported protocol version. If \node{B} accepts the version\footnote{A version is accepted by default if it is higher or equal to a nodes latest supported version. Nodes supporting experimental protocol versions might accept older versions.}, it responds with a \msg{verack} message, followed by a \msg{version} message announcing its own latest supported protocol version. Node \node{A} then decides whether it supports \node{B}'s version and sends its \msg{verack} mesage.

  If both nodes accept the connection, they both send an \msg{addr} message containing up to 1000 of its known nodes, followed by one or more \msg{inv} messages announcing all valid objects they are aware of. They then send \msg{getobject} request for all objects still missing from their inventory.

  \msg{Getobject} requests are answered by \msg{object} messages containing the requested objects.

  If a user writes a new mail on node \node{A}, it is offered via \msg{inv} to up to eight connected nodes. They will get the object and distribute it to up to eight of their connections, and so on.

  \subsection{Messages}

  The messages, objects and binary format are very well discribed in the Bitmessage wiki \cite{wiki:protocol}, the message description is therefore narrowed down to a description of what they do and when they're used.

  \subsubsection{version / verack}
  A \msg{version} message contains the latest protocol version supported by a node, as well as the streams it is interested in and which features it supports. If the other node accepts, it acknowledges with a \msg{verack} message. The connection is initialized when both nodes sent a \msg{verack} message.

  \subsubsection{addr}
  Contains up to 1000 known nodes with their IP addresses, ports, streams and supported features.

  \subsubsection{inv}
  One \msg{inv} message contains the hashes of up to 50000 valid objects. If your inventory is larger, several messages can be sent.

  \subsubsection{getdata}
  Can request up to 50000 objects by sending their hashes.

  \subsubsection{object}
  Contains one requested object, which might be one of:

  \listinginfo{}{getpubkey}{A request for a public key, which is needed to encrypt a message to a specific user.}{}
  \listinginfo{}{pubkey}{A public key. See \ref{subsec:addrenc} \nameref{subsec:addrenc}}{}
  \listinginfo{}{msg}{Content intended to be received by one user.}{}
  \listinginfo{}{broadcast}{Content sent in a way that the Addresses public key can be used to decrypt it, allowing any subscriber who knows the address to receive the such a message}{}

  \subsubsection{ping / pong}

  \subsection{Addresses \& Encryption}
  \label{subsec:addrenc}
  TODO


  \section{Issues}

  \subsection{Scalability}

  Bitmessage doen't really scale. If there are very few users, anonymity isn't given anymore, and with many users traffic and storage use grows quadratically.

  \subsubsection{Streams}
  The intended solution for this problem is splitting traffic\footnote{Addresses, actualy} into streams. When all active streams are full, a new one is created which should be used for new addresses. All users can send messages to any stream, but only listen to the streams belonging to their addresses. The unsolved problem is to determine when a stream is full. The other issue is the fact that, as the overall network grows, traffic on full streams still grows, as there are more users who might wanto to write someone on the full stream.

  \subsubsection{Prefix Filtering}
  TODO

  \subsection{Forward Secrecy}

  Obviously it's trivial for an attacker to collect all (encrypted) objects distributed through the Bitmessage network\footnote{As long as disk space is not an issue.}. If this attacker can somehow get the private key of a user, they can decrypt all stored messages intended for that user, as well as impersonate said user\footnote{The latter might be more difficult if they got the key through a brute force attack.}.

  Plausible deniability can, in some scenarios, help against this. This action, called "nuking an address", is done by anonymously publishing the private keys somewhere publicly accessible\footnote{Soo \url{https://bitmessage.ch/nuked/} for an example.}.

  Perfect forward secrecy seems impractical to implement, as it requires to exchange messages prior to sending content. That would in turn need proof of work to protect the network, resulting in twice the work for the sender and three times longer to send --- that is if both clients are online.

  \section{Discussion}

  TODO


  \bibliographystyle{plain}
  \bibliography{bibliography}

  \appendix
  \addcontentsline{toc}{section}{Appendix}
  \section*{Appendix}
  \renewcommand{\thesubsection}{\Alph{subsection}}

  \subsection{TODO}


\end{document}